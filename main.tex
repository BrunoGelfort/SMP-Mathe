\documentclass[11pt]{report}


\usepackage{amsmath}		%Matheschreibweise
\usepackage{amssymb}
\usepackage{amsfonts}
\usepackage[T1]{fontenc}
\usepackage[margin=2cm]{geometry}	%Seitenrand
\usepackage{fancyhdr}	%Titelblatt
\usepackage{graphicx}	%Bilder einfügen
\usepackage{tcolorbox}	%dicke Box
\usepackage{tikz}
\usepackage{mdframed}	%Box			%\begin{mdframed} 		und			%\end{mdframed}
\usepackage{epstopdf}
\usepackage[utf8]{inputenc}
\usepackage[ngerman]{babel}
\pagestyle{fancy}
\fancyhead{}
\fancyfoot{}
\fancyhead[L]{Skript Mathematik}
\fancyhead[R]{SMP}
\fancyfoot[C]{\thepage}

\parindent 0ex

\newenvironment{Bemerkung}{%
\begin{center}
}{%
\end{center}
}
\newenvironment{Bws}
{\par\noindent\normalfont\par\nopagebreak%
  \begin{mdframed}[
     linewidth=3pt,
     linecolor=pink,
     bottomline=false,topline=false,rightline=false,
     innerrightmargin=0pt,innertopmargin=0pt,innerbottommargin=0pt,
     innerleftmargin=1em,% Distanz zwischen vertikaler Strich & Inhalt
     skipabove=.5\baselineskip
   ]}
  {\end{mdframed}}


\newcommand{\R}{\mathbb{R}}
\newcommand{\C}{\mathbb{C}}
\newcommand{\N}{\mathbb{N}}





\begin{document}
\begin{titlepage}
	\begin{center}
	\line(1,0){300}  \\
	[0,5cm]
	\Huge{\bfseries SchülerSkript SMP} \\
	[3cm]
	\textsc{ \bfseries MATHEMATIK}  \\
	[15cm]
	\small{\today} \\
	[5cm]
	\end{center}
\end{titlepage}

\tableofcontents

\chapter{Folgen}

K"onnen wir das bitte ausprobieren?
\chapter{Reihen}

\section{g}
g\\

\chapter{Funktionsuntersuchung}

Die \textbf{Analysis} (griechisch $????????$ análysis, deutsch "`Aufl"osung"') ist ein Teilgebiet der Mathematik.  Die Untersuchung von reellen und komplexen Funktionen hinsichtlich Stetigkeit, Differenzierbarkeit und Integrierbarkeit z"ahlt zu den Hauptgegenst"anden der Analysis. Die hierzu entwickelten Methoden sind in allen Natur- und Ingenieurwissenschaften von großer Bedeutung.\\

\section{Stetigkeit und Differenzierbarkeit}

Joséphine will das machen, +Der Differentenquotient\\

\section{Ableitungsregeln}

Ein Ableitungswert gibt die Steigung an einem bestimmten Punkt an. Im Allgemeinen und zum Beweisen wird der Differentenquotient ben"otigt, um eine Ableitungsfunktion zu definieren, es geht aber in vielen F"allen schneller.

\subsection{Produktregel}

Sind die Funktionen $u$ und $v$ an der Stelle $x_{0}$ $\in$ $D$ differenzierbar, dann ist die Funktion $f(x)=u(x)\cdot v(x)$ bei $x_{0}$ auch differenzierbar und es gilt: \\
\\
$f'(x0) = u'(x_{0})v(x_{0})+u(x_{0})v'(x_{0})$\\


\subsection{Quotientenregel}

Sind die Funktionen $u$ und $v$ an der Stelle $x_{0}$ $\in$ $D$ differenzierbar, dann ist die Funktion $f(x)=\frac{u(x)} {v(x)}$ bei $x_{0}$ auch differenzierbar und es gilt: \\
\\
$f'(x_{0})=\frac{u'(x_{0})\cdot v(x_{0})-u(x_{0})\cdot v'(x_{0})}{v^2(x_{0})}$


\subsection{Kettenregel}

Die Funktion $v$ sei an der Stelle $x_{0}$ differenzierbar und die Funktion $u$ an der Stelle $v(x_{0})$. Dann ist die Funktion $f=u\circ v$ mit der Gleichung $f(x)= u(v(x))$ an der Stelle $x_{0}$ differenzierbar. Es gilt:

$f'(x_{0})=v'(x_{0})\cdot u'(v(x_{0}))$


\subsection{Tangente und Normale}

Ist die Funktion $f$ differenzierbar an der Stelle $x_{0}$, dann hat die \textbf{Tangente} an dem Graphen von $f$ die Steigung $a=f'(x_{0})$ und den Y-Achsenabschnitt $b=-f'(x_{0})\cdot x_{0}+f(x_{0})$. Daraus ergibt sich die Tangentengleichung:\\
\\
$T_{x_{0}}(x)=f'(x_{0})\cdot (x-x_{0})+f(x_{0}) $ \\
\\
Eine Merkhilfe dazu ist das Wort "`Fuxufu"', wobei "`u"' dem $x_{0}$ entspricht.\\
\\
Die \textbf{Normale} an der Stelle $x_{0}$ bezeichnet die Gerade, die genau senkrecht zur Tangente steht und diese im Ber"uhrpunkt des Graphen schneidet.\\
\\
$N_{x_{0}}(x)=\frac{1}{f'(x_{0})}\cdot (x-x_{0})+f(x_{0})$\\


\section{Vollst"andige Funktionsuntersuchung}


\subsection{Definitionsbereich}

Am Anfang muss der Definitionsbereich angegeben werden, um eventuelle Divisionen durch null zu vermeiden. \\


\subsection{Achsenschnittpunkte}

Es gibt zwei Arten von Achsenschnittpunkten:
\begin{enumerate}
\item X-Achsenschittpunkte (Nullstellen), die man mit $f(x)=0$ herausfindet
\item Y-Achsenschnittpunkt, den man durch einsetzen bekommt: $f(0)$ \\
\end{enumerate}

\subsection{Symmetrie}

\subsubsection{Y-Achsensymmetrie}

Durch L"osung der Gleichung $f(x)=f(-x)$ findet man heraus ob die Funktion achsensymmetrisch ist.\\
Zudem ist die Funktion dann achsensymmetrisch, wenn nur gerade Exponenten vorhanden sind.\\

\subsubsection{Symmetrie zum Origo}

Durch L"osung der Gleichung $f(x)=-f(-x)$ findet man heraus ob die Funktion punktsymmetrisch ist.\\
Zudem ist die Funktion dann punktsymmetrisch, wenn nur ungerade Exponenten vorhanden sind.\\


\subsubsection{Symmetrie zu einem beliebigen Punkt}

Symmetrie zu einem Punkt liegt vor, wenn f"ur den Punkt $P(x_{0}|y_{0})$ gilt:\\
\\
$f(x_{0}+h)-y_{0}=-f(x_{0}-h)+y_{0}$

\begin{Bemerkung}
Beispiel\\
\end{Bemerkung}
$f(x)=\frac{x}{x-1}$\\
\\
Aus dem Schnittpunkt der Asymptoten kann man vermuten, dass $f(x)$ achsensymmetrisch zum Punkt $P(1|1)$ ist.\\
\\
$\left. \begin{array}{rcl}
\Rightarrow f(x_{0}+h)-y_{0}=\frac{1+h}{1+h-1}-1=\frac{1}{h}\\
\\
\Rightarrow -f(x_{0}-h)+y_{0}=-[\frac{1-h}{1-h-1}+1]=\frac{1}{h}\\
\end{array}\right\} \frac{1}{h}=\frac{1}{h}$ Die Funktion $f$ ist zu $P$ symmetrisch.\\



\subsection{Grenzwerte}



\subsection{Monotonie}

\subsection{Extremstellen}

\subsection{Wendestellen}

\subsection{Beispiel}

	\section{Funktionenscharen}


Erkl"arung:\\

\begin{minipage}[b]{0.5\linewidth}
Eine \textbf{Funktionenschar} ist eine Menge von Funktionen, die neben der Variable auch noch einen ver"anderlichen Parameter im Funktionsterm enth"alt. Jedem Wert des Parameters ist ein Graph der Schar zugeordnet. Der Parameter, oft $a$, wird hierbei "uberall wie eine Konstante behandelt.\\
Der Punkt, den alle Graphen, unabh"angig von ihren Parametern, beinhalten, nennt man B"undel. Die Graphen einer
Funktionenschar bilden gemeinsam eine Kurvenschar.\\
Hier ist die Kurvenschar der Funktion\\ $f(x)=ax^3$. Sie verlaufen alle durch das B"undel $P(0|0)$
\end{minipage}
\hfill
\begin{minipage}[b]{0.4\linewidth}
\includegraphics[height=12\baselineskip]{BundelFunktionenscharen.eps}
\end{minipage}

\chapter{Trigonometrie}

		\begin{tikzpicture}[scale=5.3,cap=round,>=latex]
		        % draw the coordinates
		        \draw[->] (-1.5cm,0cm) -- (1.5cm,0cm) node[right,fill=white] {$x$};
		        \draw[->] (0cm,-1.5cm) -- (0cm,1.5cm) node[above,fill=white] {$y$};

		        % draw the unit circle
		        \draw[thick] (0cm,0cm) circle(1cm);

		        \foreach \x in {0,30,...,360} {
		                % lines from center to point
		                \draw[gray] (0cm,0cm) -- (\x:1cm);
		                % dots at each point
		                \filldraw[black] (\x:1cm) circle(0.4pt);
		                % draw each angle in degrees
		                \draw (\x:0.6cm) node[fill=white] {$\x^\circ$};
		        }

		        % draw each angle in radians
		        \foreach \x/\xtext in {
		            30/\frac{\pi}{6},
		            45/\frac{\pi}{4},
		            60/\frac{\pi}{3},
		            90/\frac{\pi}{2},
		            120/\frac{2\pi}{3},
		            135/\frac{3\pi}{4},
		            150/\frac{5\pi}{6},
		            180/\pi,
		            210/\frac{7\pi}{6},
		            225/\frac{5\pi}{4},
		            240/\frac{4\pi}{3},
		            270/\frac{3\pi}{2},
		            300/\frac{5\pi}{3},
		            315/\frac{7\pi}{4},
		            330/\frac{11\pi}{6},
		            360/2\pi}
		                \draw (\x:0.85cm) node[fill=white] {$\xtext$};

		        \foreach \x/\xtext/\y in {
		            % the coordinates for the first quadrant
		            30/\frac{\sqrt{3}}{2}/\frac{1}{2},
		            45/\frac{\sqrt{2}}{2}/\frac{\sqrt{2}}{2},
		            60/\frac{1}{2}/\frac{\sqrt{3}}{2},
		            % the coordinates for the second quadrant
		            150/-\frac{\sqrt{3}}{2}/\frac{1}{2},
		            135/-\frac{\sqrt{2}}{2}/\frac{\sqrt{2}}{2},
		            120/-\frac{1}{2}/\frac{\sqrt{3}}{2},
		            % the coordinates for the third quadrant
		            210/-\frac{\sqrt{3}}{2}/-\frac{1}{2},
		            225/-\frac{\sqrt{2}}{2}/-\frac{\sqrt{2}}{2},
		            240/-\frac{1}{2}/-\frac{\sqrt{3}}{2},
		            % the coordinates for the fourth quadrant
		            330/\frac{\sqrt{3}}{2}/-\frac{1}{2},
		            315/\frac{\sqrt{2}}{2}/-\fra
								c{\sqrt{2}}{2},
		            300/\frac{1}{2}/-\frac{\sqrt{3}}{2}}
		                \draw (\x:1.25cm) node[fill=white] {$\left(\xtext,\y\right)$};

		        % draw the horizontal and vertical coordinates
		        % the placement is better this way
		        \draw (-1.25cm,0cm) node[above=1pt] {$(-1,0)$}
		              (1.25cm,0cm)  node[above=1pt] {$(1,0)$}
		              (0cm,-1.25cm) node[fill=white] {$(0,-1)$}
		              (0cm,1.25cm)  node[fill=white] {$(0,1)$};
		    \end{tikzpicture}

\chapter{Vektorielle Geometrie}
dsdfs\\


\section{g}
f\\

\chapter{Komplexe Zahlen}



	\section{Einf"uhrung}
\underline{Problem:} Es gibt algebraische Gleichungen, die in der Menge  der reellen Zahlen $\R$  keine L"osung besitzen.\\
\\
$\Rightarrow x^2 + 1 = 0   $\\
$\Leftrightarrow x = ± \sqrt{-1}$ Keine Reelle L"osung!\\
\\
Es kann hierbei ein neues Symbol eingeführt werden : $i = \sqrt{-1} $\\
Damit kann man der obigen Gleichung die Lösung $x =±i$ zuordnen\\
\\
Wenn wir voraussetzen, dass diese neue Zahlen nach denselben Rechengesetzen genügen, wie die reellen Zahlen, erhalten wir damit auch Lösungen für andere bisher nicht lösbare quadratische Gleichungen, wie das folgende Beispiel zeigt:\\
\\
$\Rightarrow x^2 + 2x + 5 = 0$\\
\\
$\Rightarrow x = \dfrac { -2± \overbrace{ \sqrt{4-20}}^{=\sqrt{16*(-1)}} } { 2}
=\frac {-2 ±4i}{2}
= -1±2i$\\
\\
\\
\begin{Bemerkung}
Bezeichnungen\\
\end{Bemerkung}

\begin{enumerate}
\item Der Ausdruck $\sqrt{-1}$ heißt \textbf{imagin"are Einheit} und wird hier mit $i$ bezeichnet.
\item Ausdr"ucke der Form $i\cdot y$ mit $y \in \R$  heißen \textbf{imagin"are Zahlen}
\item Ausdr"ucke der Form $z =x+i*y$ mit $x,y \in \R$ werden als \textbf{Komplexe Zahlen} bezeichnet
\item Ist $z =x+i*y$ eine Komplexe Zahl, so heißen\\
\indent $x=Re(z)$ \textbf{Realteil} von $z$\\
\indent $y=Im(z)$ \textbf{Imagin"arteil} von $z$
\item Die Menge $\C =\{ {z = x+j y| x, y \in \R}\}$ wird als Menge der Komplexen Zahlen bezeichnet\\
\end{enumerate}

\begin{Bemerkung}
Aber\\
\end{Bemerkung}

Der  \textbf{Imagin"arteil} $y$ einer komplexen Zahl  $z =x+i*y$ ist selbst eine reelle Zahl! Der \textbf{Imagin"arteil} ist lediglich der Faktor bei $i$!\\

	\section{Darstellung komplexer Zahlen}

Eine komplexe Zahl wird durch zwei reelle Zahlen charakterisiert. Wie bei zweidimensionalen Vektoren brauchen wir hier zur geometrischen Veranschaulichung auch eine zweidimensionale Ebene.\\

	\subsection{Kartesische Darstellung}

Jeder komplexen Zahl $z =x+i*y$ entspricht genau ein Punkt $P =(x,y)$ in der komplexen Zahlenebene und umgekehrt.\\

\includegraphics[width=3in]{komplexezahlen1}

\begin{Bemerkung}
Bezeichnungen\\
\end{Bemerkung}

\begin{enumerate}
\item Die komplexe Zahlenebene nennt sich auch Gaußsche- Zahlenebene
\item Hier werden die Achsen des Koordinatensystems als \textbf{reelle Achse} bzw. \textbf{imagin"are Achse} bezeichnet.\\
\end{enumerate}

\begin{Bemerkung}
Beispiel\\
\end{Bemerkung}

Die folgenden komplexen Zahlen sind in der Gaußschen Zahlenebene darzustellen: \\
$z_{1} = 2+3*j$  $z_{2} =-3-j$ ($i$ wird hier $j$ genannt)

\includegraphics[width=3in]{komplexezahlen2}

\begin{Bemerkung}
Bemerkungen\\
\end{Bemerkung}

F"ur manche Anwendungen ist es hilfreich, eine komplexe Zahl nicht als Punkt $P=(x,y)$ in der Gaußschen Zahlenebene zu veranschaulichen, sondern stattdessen den Ortsvektor zu betrachten

$z=x+j*y \Leftrightarrow z=
\begin{pmatrix}
x\\
y\\
\end{pmatrix}$

\includegraphics[width=3in]{komplexezahlen3}

In diesem Fall spricht man von z als einem \textbf{komplexen Zeiger}.\\

	\subsection{Polarkoordinatendarstellung}

Neben der eben eingef"uhrten kartesischen Darstellung $z =x+j*y$ kann eine komplexe Zahl auch entsprechend der hier stehenden Skizze dich ihren Radius  und den Winkel  eindeutig festgelegt werden.

\includegraphics[width=3in]{komplexezahlen4}

\begin{Bemerkung}
Erinnerung\\
\end{Bemerkung}

Zusammenhang zwischen den Koordinaten $P(x,y)$ und $P(r,\varphi)$ :\\

$
\begin{pmatrix}
x=r*cos(\varphi)\\
\\
y=r*sin(\varphi)
\end{pmatrix}
$
bzw.
$
\begin{pmatrix}
r= \sqrt{x^2+y^2}\\
\\
tan(\varphi) = \frac{y}{x}
\end{pmatrix}
$
\\
\\

\begin{Bemerkung}
Bemerkung\\
\end{Bemerkung}

Der Zusammenhang zwischen dem Quotienten $\frac{x}{y}$ und dem Winkel $\varphi \in [0,2\pi)$ eindeutig, da die Tangensfunktion -periodisch ist.\\
Damit erhält man die \textbf{trigonometrische Darstellung} :\\

$z=x+j*y =r*cos(\varphi)+j*r*sin(\varphi) \Rightarrow z=r(cos(\varphi)+j*sin(\varphi))$\\

Dieser Ausdruck von $z$ wird im Folgenden sehr h"aufig auftreten. Deshalb wird daf"ur die Abk"urzung\\
\\
$e^{j\varphi} =cos(\varphi)+j*sin(\varphi)$\\
\\
ein.
Somit ergibt sich schließlich eine sehr kompakte Darstellung, die sogenannte \textbf{Exponentialdarstellung} einer komplexen Zahl:\\\\
$z=  {r(cos(\varphi)+r*j*sin(\varphi))}=r*e^{j\varphi}$

\begin{Bemerkung}
Zusammenfassung\\
\end{Bemerkung}

Eine komplexe Zahl l"asst sich auf verschiedene Arten darstellen:\\


\begin{tcolorbox}

\begin{enumerate}

\item $z=x+j*y$ (kartesische Darstellung)
\item $z=(cos(\varphi)+j*sin(\varphi))$ (trigonometrische Darstellung)
\item $z=r*e^{j\varphi}$ (Exponential-Darstellung)\\

\end{enumerate}

\end{tcolorbox}

	\subsection{Umrechnung zwischen den Darstellungen}

\chapter{Statistik und Wahrscheinlichkeit}
dsfdfs\\


	\section{g}
fhdshfdsgfds\\

\chapter{Arithmetik}
fsdfsdiutzuizfjfgh\\

\chapter{Arithmetik}
fsdfsdiutzuizfjfgh\\

\chapter{Matrizen}

	\section{Lineare Gleichungssysteme und Gauß-Algorythmus}

Lineare Gleichungssysteme lassen sich aufwendig mit Einsetzungsverfahren oder Additionsverfahren l"osen, Carl Friedrich Gauß hat ein ''Algorythmus'' erfunden, mit dem sich ohne Taschenrechner leicht und relativ schnell l"osen l"asst.\\
Am Besten wird dieser mit einem Beispiel Erl"autert:\\
\\
$\Rightarrow$ $\left\{ \begin{array}{rccl}
4x+3y+z&=&13& (1)\\
2x-5y+3z& =& 1 &(2)\\
7x-y-2z&=&-1&(3)\\
\end{array}\right.$ \qquad \{ $1\cdot (1) -2\cdot (2)$ \}  und \{ $7\cdot (1) -4\cdot (3)$ \}  \qquad \qquad $D=\R^3$\\
\\
\\
Hier versucht man in Zeile (2) und (3) die erste Variabel zu eliminieren\\
\\
$\Leftrightarrow$ $\left\{ \begin{array}{rcl}
4x+3y+z&=&13\\
0x +13y -5z&=& 11\\
0x +25y +15z&=& 95\\
\end{array}\right.$ \qquad \{ $25\cdot(2) -13\cdot(3)$ \}  \\
\\
\\
Jetzt versucht man die zweite Variabel in der dritten Gleichung zu eliminieren\\
\\
$\Leftrightarrow$ $\left\{ \begin{array}{rcl}
4x+3y+z&=&13\\
0x +13y -5z&=& 11\\
0x+0y-320z&=&-960 $\qquad$ \Leftrightarrow z=3 \\
\end{array}\right.$\\
\\
\\
Jetzt wird eingesetzt\\
\\
$\Leftrightarrow$ $\left\{ \begin{array}{rcl}
4x+3y+z&=&13\\
0x+13y-5\cdot3&=&11 $\qquad$ \Leftrightarrow y=2\\
0x+0y+z&=&3\\
\end{array}\right.$\\
\\
\\
$\Leftrightarrow$ $\left\{ \begin{array}{rcl}
x+0y+0z&=&1\\
0x+y+0z&=&2\\
0x+0y+z&=&3\\
\end{array}\right.$\\
\\
\\
$\mathbb{L}=\{(1|2|3) \}$ Die Lösungsmenge wird als n-Tupel alphabetisch sortiert.

	\section{LGS mit dem Taschenrechner l"osen}

\chapter{Algorythmik}
sdffzgkzgs\\


	\section{g}
fljzgkg\\

%und so weiter



\end{document}
