\chapter{Matrizen}

	\section{Lineare Gleichungssysteme und Gauß-Algorythmus}

Lineare Gleichungssysteme lassen sich aufwendig mit Einsetzungsverfahren oder Additionsverfahren l"osen, Carl Friedrich Gauß hat ein ''Algorythmus'' erfunden, mit dem sich ohne Taschenrechner leicht und relativ schnell l"osen l"asst.\\
Am Besten wird dieser mit einem Beispiel Erl"autert:\\
\\
$\Rightarrow$ $\left\{ \begin{array}{rccl}
4x+3y+z&=&13& (1)\\
2x-5y+3z& =& 1 &(2)\\
7x-y-2z&=&-1&(3)\\
\end{array}\right.$ \qquad \{ $1\cdot (1) -2\cdot (2)$ \}  und \{ $7\cdot (1) -4\cdot (3)$ \}  \qquad \qquad $D=\R^3$\\
\\
\\
Hier versucht man in Zeile (2) und (3) die erste Variabel zu eliminieren\\
\\
$\Leftrightarrow$ $\left\{ \begin{array}{rcl}
4x+3y+z&=&13\\
0x +13y -5z&=& 11\\
0x +25y +15z&=& 95\\
\end{array}\right.$ \qquad \{ $25\cdot(2) -13\cdot(3)$ \}  \\
\\
\\
Jetzt versucht man die zweite Variabel in der dritten Gleichung zu eliminieren\\
\\
$\Leftrightarrow$ $\left\{ \begin{array}{rcl}
4x+3y+z&=&13\\
0x +13y -5z&=& 11\\
0x+0y-320z&=&-960 $\qquad$ \Leftrightarrow z=3 \\
\end{array}\right.$\\
\\
\\
Jetzt wird eingesetzt\\
\\
$\Leftrightarrow$ $\left\{ \begin{array}{rcl}
4x+3y+z&=&13\\
0x+13y-5\cdot3&=&11 $\qquad$ \Leftrightarrow y=2\\
0x+0y+z&=&3\\
\end{array}\right.$\\
\\
\\
$\Leftrightarrow$ $\left\{ \begin{array}{rcl}
x+0y+0z&=&1\\
0x+y+0z&=&2\\
0x+0y+z&=&3\\
\end{array}\right.$\\
\\
\\
$\mathbb{L}=\{(1|2|3) \}$ Die Lösungsmenge wird als n-Tupel alphabetisch sortiert.

	\section{LGS mit dem Taschenrechner l"osen}
